\section{Conclusiones}

\subsection{John Sebastián Galindo Hernández}
La red neuronal LeNet es un paso gigante de las redes neuronales artificiales 
a las redes neuronales convolucionales, ya que estas últimas son capaces de
reconocer patrones en imágenes, lo cual es una tarea muy compleja y más aún
cuando se tienen muchas categorías a clasificar. La red LeNet es un ejemplo de
una CNN implementada con el objetivo de mantener una generalización en el 
entrenamiento mientras se intenta reducir la información de las imágenes 
originales solo a los patrones más importantes, esto se logra mediante las 
capas de su arquitectura. Adentrandose en la opinion sobre como predice la red
LeNet, se puede decir que es muy buena, ya que en la mayoría de los casos
predice correctamente la categoría de la imagen, sin embargo, en algunos casos
que parecen sencillos para un humano, la red falla, esto se puede deber a que 
los datos de entrenamiento no son suficientes o algunas imagenes tienen un 
ruido que termina confundiendo a la red. En general, la red LeNet es una
herramienta muy útil para clasificar imágenes, pero se debe tener en cuenta que
al trabajar con imágenes en tiempo real se debe tener presente un estandar en
posicionamiento, tamaño y color de las numeros en las imágenes para que la red pueda
clasificar correctamente. Por último, LeNet es una red que obtiene muy bien las 
características y patrones de los datos de entrada, lo cual la hace maleable para
diferentes casos como el de las señales de trafico, las imagenes de cancer de mama,
las imagenes de letras, números y simbolos del codigo ASCII
o las imagenes de multiples dígitos escritos a mano, en esta ultima, también se 
puede comprender que la red puede ser juntada con otras técnicas que segmenten 
la información en una imagen para separar los dígitos y clasificarlos de manera
independiente dando así una clasificación más precisa y que se pueda volver a 
combinar para obtener el número completo.

\subsection{Miguel Ángel Moreno Beltrán}
La red neuronal convolucional LeNet destaca por su relevancia en entornos actuales, 
gracias a su capacidad para generalizar patrones visuales. 
Aunque es limitada para tareas más complejas, su precisión puede mejorarse mediante técnicas avanzadas, 
como se observó con el desarrollo de LeNet-5 y su versión mejorada Boosted LeNet-4. 
En este laboratorio, el uso de funciones de activación modernas como ReLU y 
el optimizador Adam permitieron alcanzar un rendimiento superior del 93.1\%, 
en comparación con las configuraciones originales. 
Para finalizar, LeNet demuestra un rendimiento altamente competitivo en entornos 
específicos y controlados, como el reconocimiento de números en imágenes, 
manteniéndose como una arquitectura eficaz y adaptable a diversas aplicaciones.


\subsection{Conclusion Final}
LeNet ha demostrado ser una red neuronal convolucional pionera y relevante en el ámbito actual, 
gracias a su capacidad para reconocer patrones visuales y clasificar imágenes de forma precisa. 
La arquitectura de LeNet, especialmente en su versión LeNet-5, 
fue un avance significativo hacia el desarrollo de redes neuronales más complejas y adaptadas a 
tareas de procesamiento de imágenes. Las mejoras, como el Boosted LeNet-4, y el uso de técnicas 
modernas como las funciones de activación ReLU y el optimizador Adam, han mostrado que su rendimiento 
puede alcanzar niveles competitivos, con una precisión superior al 93.1\% en condiciones controladas. 
Aunque limitada para algunas tareas complejas y entornos no controlados, 
LeNet sigue siendo una herramienta útil y adaptable a diversas aplicaciones, 
desde el reconocimiento de números y letras hasta el procesamiento de señales de tráfico y la detección 
de anomalías en imágenes médicas. Su capacidad para extraer y generalizar patrones la convierte 
en una opción ideal para proyectos específicos, donde con el apoyo de técnicas complementarias, 
como la segmentación de imágenes, podría mejorar aún más su precisión y aplicabilidad en la clasificación 
de imágenes complejas y ruidosas.